%************************************************
\chapter{Untersuchung geeigneter Mittel}\label{ch:tools} 
%************************************************

Wie in Kapitel \ref{ch:php7} gezeigt, sind die Veränderungen zwischen \acs{PHP} 5 und \acs{PHP} 7 nicht nur sehr umfangreich, sondern erfordern auch große Eingriffe 
in den betroffenen Quellcode. \textbf{ISO/IEC 14764} sieht vor, die Migration zu definieren

\section{Erkennung des zu ändernden Codes}
    Um alten Code migrieren zu können, müssen alle Stellen gefunden werden, die in ihrer ursprünglichen Form in der neuen Umgebung nicht lauffähig sind. Dafür relevante
    Beispiele sind in Kapitel \ref{ch:php7} gelistet, die gesamte Liste kann der Dokumentation entnommen werden. Die Erkennung kann je nach Umfang des Quellcodes
    und der verwendeten Funktionen entweder manuell oder automatisiert durchgeführt werden. Für beide Arten werden im folgenden Beispiele genannt und die jeweiligen
    Vor- und Nachteile diskutiert.
    \subsection{Manuelle Erkennung}
        Eine manuelle Erkennung des Codes bietet sich vor allem bei kleinen Softwareprojekten an, bei denen ein vollumfänglicher Überblick über den eingesetzten Code
        besteht. Hier kann durch die in typischen Editoren und Entwicklungsumgebungen integrierte Suche genutzt werden um alle Vorkommen von nicht lauffähigen
        Funktionen zu finden und diese anschließend einem Refactoring zu unterziehen. Besonders einfach gestaltet sich diese Methode bei entfernten Funktionen,
        beispielsweise die der Erweiterung \textit{ereg}\ref{ereg}. Diese kann der Entwickler in der Dokumentation nachschlagen und den Code auf etwaige Vorkommen prüfen.
        Schwierig wird die manuelle Erkennung bei Änderungen wie der Einhaltung des Standards in Switch-Anweisungen\ref{switch}. Hier ist eine Suche nur über umfangreiche
        Suchmuster (Reguläre Ausdrücke) möglich, die meist nicht trivial zu erstellen sind und viele Einzelfälle (z.B. verschachtelte Switch-Anweisungen) abdecken müssen.
        In diesen Fällen ist durch die manuelle Suche höchstens eine Eingrenzung des Problems möglich.
    \subsection{Automatisierte Erkennung}
        Da die zuvor besprochene manuelle Erkennung betroffenen Codes nur für einzelne Fälle oder kleine Projekte in Frage kommt, bietet sich als alternative die 
        automatiche Erkennung an, mit dem Ziel, dem Entwickler einen vollumfänglichen Überblick der zu überarbeitenden Stellen im Code zu liefern. Im vorliegenden Fall
        wurde das Tool \textit{\ac{php7mar}}\footnote{Alexia. php7mar. URL: \url{https://github.com/Alexia/php7mar}} des Entwicklers \textit{Alexia} genutzt.
        \textit{\ac{php7mar}} erkennt mithilfe von \textbf{Regulären Audrücken}, \textbf{String-Matching} und \textbf{Lexikalischer Analyse} kritischen Code in 
        Projekten und generiert daraus einen Bericht, bestehend aus Zeilenangaben, gefundenen Problemen und Lösungshinweisen. Ein Beispiel eines solchen Berichts findet
        sich in Listing\ref{lst:php7mar}. In der Datei \textit{GMCSS.php} werden drei Fehlerklassen gefunden: Erstens mehrere Fälle der Nutzung der veralteten Definition von
        Konstruktoren, zweitens einige Vorkommen der Entfernten Erweiterung \textit{mysql}, sowie drittens ein indirekter Variablenzugriff, dessen Aussage unter PHP 7
        möglicherweise eine andere ist (vgl. Kapitel~\ref{indirectVar}). Insbesondere der erste Fall zeigt die Überlegenheit eines Analysetools, da solche Fehler nur
        schwer mit einer trivialen Suche zu finden sind.

        \begin{lstlisting}[caption={Beispiel eines generierten Berichts mit \textit{\ac{php7mar}}}, label={lst:php7mar}]
            #### C:\Users\Nutzer\Documents\GitHub\gambio_tickets75\StyleEdit\classes\GMCSS.php
            * oldClassConstructors
            * Line 55:  `function GMCSS($p_css_file, $p_type='archive')`
            * Line 384: `function GMCSSImport($p_css_file = false, $p_import_mode = '')`
            * Line 791: `function GMCSSExport($p_css_file)`
            * Line 912: `function GMCSSUpload($p_files, $p_type)`
            * Line 982: `function GMCSSArchive()`
            * deprecatedFunctions
            * Line 302: `$t_css_query = mysql_query("`
            * Line 311: `if((int)mysql_num_rows($t_css_query) > 0)`
            * Line 313: `$t_row_styles = mysql_fetch_array($t_css_query, MYSQL_ASSOC);`
            * Line 316: `$t_css_query = mysql_query("`
            * Line 325: `if((int)mysql_num_rows($t_css_query) > 0)`
            * variableInterpolation
            * Line 359: `global $$shippingModule; //notice $$`
        \end{lstlisting}

\section{Refactoring}
    \subsection{Unit-Tests}
    \subsection{Search \& Replace}
    \subsection{Wrapping}

\section{Lauffähigkeit historischen Codes}
    \subsection{Codeverwaltung}
    \subsection{Lokale Entwicklungsumgebung}
    \subsection{Continous Integration mittels Containern}


