\chapter{Einleitung}
Am 03. Dezember 2015 erschien mit PHP 7.0.0 das erste Major-Release der Programmiersprache seit elf Jahren. Damit einhergehend wurde die Einstellung der Weiterentwicklung der vorhergehenden Version
5 für den 10. Januar 2019 angekündigt. Der Entwicklungsstopp führt dazu, dass Sicherheitslücken in der Implementation der alten Version nicht mehr geschlossen werden, was 
wiederum dazu führt, dass bereits ausgelieferte Software angreifbar wird sobald neue Lücken gefunden werden. \\
Derzeit setzen 79,1\% der 10 Millionen meistgenutzten Webseiten PHP als serverseitige Programmiersprache ein, davon 61,5\% PHP in der veralteten Version 5\footnote{W3Techs, \glqq Usage statistics of PHP for websites\grqq , 
\url{https://w3techs.com/technologies/details/pl-php/all/all}}. Diese Installationen können allesamt als unsicher eingestuft werden. Seit der letzten Veröffentlichung 
unter Version 5 wurden vier neue Schwachstellen veröffentlicht\footnote{CVE details, \glqq PHP 5.6.40 Security Vulnerabilities\grqq , 
\url{https://www.cvedetails.com/vulnerability-list/vendor_id-74/product_id-128/version_id-298516/PHP-PHP-5.6.40.html}}, die in unterstützten Versionen bereits geschlossen 
wurden.

\section{Motivation}
Die Firma \textit{tickets75}, eine unabhängige Ticketagentur, die sich auf die Vermittlung von Tickets für begehrte 
Veranstaltungen über den eigenen Onlineshop spezialisiert hat. Als e-Commerce-Unternehmen ist der reibungslose Betrieb der 
Onlinepräsenz besonders wichtig. Ebenso hat die Sicherheit von Kunden, insbesondere in Bezug auf deren Zahlungsmittel oberste 
Priorität. Um diese Sicherheit weiterhin gewährleisten zu können, soll der Onlineshop für die Ausführung unter \ac{PHP} optimiert 
werden. Der Shop basiert auf der quelloffenen e-Commerce-Plattform \textit{Gambio}, wurde in der Vergangenheit jedoch stark 
angepasst, sodass eine einfache Aktualisierung des Grundsystems nicht mehr in betracht gezogen werden kann.

\section{Aufgabenstellung}
Ziel der Arbeit ist die Migration eines Onlineshops von PHP 5.6 zu PHP 7. An diesem Beispiel sollen Teschniken und Technologien 
gezeigt werden, die ein Upgrade der Programmiersprache einfacher gestalten, oder erst in effizienter Weise ermöglichen. 
Gleichzeitig werden die Ziele des Major Release von PHP herausgearbeitet und anhand des praktischen Beispiels überprüft.
Als Leitfaden für die Migration des Onlineshops dient dabei der Internationale Standard \textbf{ISO/IEC 14764}.

\section{Aufbau}
Kapitel 2 beleuchtet die Grundlagen, dieser Arbeit zugrunde liegen. Dazu wird die Entwicklung der Programmiersprache PHP 
beleuchtet, sowie auf die Hintergründe von Wartung und Versionierung von Software eingegangen.
Kapitel 3 zeigt die Änderungen der PHP-API von Version 5.6 zu 7.0 anhand von Beispielen und zeigt dadurch die Ziele in der 
Entwicklung der neuen Version auf.
Kapitel 4 stellt die Werkzeuge zur Vorbereitung und Durchführung der Migration vor und diskutiert Vor- und Nachteile verschiedener 
Optionen.
Kapitel 5 geht auf die Ausführung der Migration ein. Dazu wird eine Analyse des Quellcodes vorgenommen, deren Ergenbisse 
zur Migration genutzt werden. 
In Kapitel 6 werden die Ergebnisse der Migration ausgewertet, insbesondere nach dem Kriterium der Ausführungsgeschwindigkeit.
Kapitel 7 zeigt Schlussbetrachtungen und gibt einen Ausblick auf mögliche weitere Arbeitsthemen.
%TODO