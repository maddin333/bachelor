%************************************************
\chapter{Auswertung}\label{ch:end} 
%************************************************

Die vorangegangenen Kapitel zeigen, dass die Migration einer komplexen Software keineswegs trivial ist. Gleichzeitig vereinfachen verschiedene Technologien die 
sowohl die Arbeit, als auch die Beachtung des Standards \textbf{ISO/IEC 14764}. Die Anforderungsanalyse in Kapitel~\ref{analyze} zeigt, dass eine manuelle Erkennung 
des zu ändernden Codes aufgrund der Menge nahezu unmöglich ist und nur mit technischen Hilfsmitteln wie dem in Kapitel~\ref{automatic} vorgestellten \textit{php7mar} 
zu überblicken ist. Dazu ist jedoch auch zu erwähnen, dass diese Hilfsmittel unter Umständen fehlerhaft sind oder nicht alle Bereiche der Migration abdecken und 
dementsprechend erst (weiter-) entwickelt werden müssen. Die in \textbf{ISO/IEC 14764} geforderte Unterstützung des historischen Codes kann zwar auf klassischem Wege 
gelöst werden, stellt sich jedoch, wie in Kapitel~\ref{VCS} dargelegt, für umfangreiche Projekte nur unter Verwendung von Virtualisierung als sinnvoll dar. 
Die in Kapitel~\ref{develop} gezeigten Änderungen an der Software zeigen, wie wichtig die aufgezeigten Techniken des Refactorings für die Migration einer Software sind.
Zudem zeigen diese Änderungen, dass zwei der in Kapitel~\ref{ch:php7} herausgearbeiteteten Ziele des neuen Major-Release von \ac{PHP} erreicht wurden: \\
So zeigen beispielsweise die Kapitel~\ref{indirect} und \ref{construct}, dass Sonderfälle reduziert wurden und Sprachkonstrukte in \ac{PHP} vereinheitlicht wurden. 
Das Ziel der erhöhten Sicherheit von Software wird über die Entfernung veralteter Erweiterungen (vgl. Kapitel~\ref{mysql}) sowie potentiell gefährlicher Optionen (vgl. Kapitel~\ref{preg}) 
erreicht. Das letzte Ziel, die Erhöhung der Ausführungsgeschwindigkeit, lässt sich über das in Kapitel~\ref{newrelic} vorgestellte Tool \textit{New Relic} überprüfen. 
Eine Auswertung der 24 Stunden vor, sowie nach der Migration (dargstellt in Abbildung~\ref{fig:time}) zeigt, dass die durchschnittliche Ausführungszeit von Anfragen an den Onlineshop 
von 66,05ms auf 43,38ms und somit um ca. 34\% zurückgegangen ist.

\begin{figure}[bth]
    \myfloatalign
    {\includegraphics[width=1\linewidth]{gfx/chart}} \quad
    \caption[Ausführungszeiten von Anfragen vor und nach der Migration]{Ausführungszeiten von Anfragen vor und nach der Migration}\label{fig:time}
\end{figure}


\chapter{Schlussbetrachtungen}
 
