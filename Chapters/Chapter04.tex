%************************************************
\chapter{Migration des Tickets75 Onlineshops}\label{ch:migration} 
%************************************************

Dieses Kapitel beschreibt die Migration des Onlineshops der Firma \textit{tickets75} von \ac{PHP} 5.6 zu \ac{PHP} 7.3, sowie die 
genutzten Werkzeuge und Techniken. Die Codebasis 
für das Projekt umfasst 5732 einzelne \ac{PHP}-Dateien, bestehend aus 596.198 Zeilen Code. Diese Menge verdeutlicht, dass 
entsprechende Werkzeuge zur Automatisierung nötig sind, da kein Überblick über die Gesamtheit der Codebasis bestehen kann.

\section{Anforderungsanalyse}
Die Anforderungsanalyse ist der wichtigste Schritt zur Vorbereitung der Migration. Durch sie kann eine Abschätzung getroffen werden, 
wie schwer und welche Teile des Codes von der Migration betroffen sind. Dies ist vor allem wichtig, hinsichtlich der 
Wirtschaftlichkeit einer Migration. Ist der Anteil zu migrierenden Codes zu hoch, kann es sinnvoll sein, eine Software 
komlett neu zu schreiben. Im vorliegenden Fall wurde die Analyse mit dem Programm \textit{php7mar} durchgeführt.
Die Ergebnisse, dargestellt in \ref{tab:migrationPercentage}, zeigen, dass zwar über 10\% der Dateien unter \ac{PHP} 7 
Fehler enthalten, gemessen an den betroffenen Codezeilen aber nur ein kleiner Teil (0,24\%) des Codes migiriert werden muss.
Eine granularere Analyse erlaubt das angehängte Python-Skript, welches die tatsächlich genutzten Funktionen aus dem Report 
herausfiltert. Die Ergebnisse in Tabelle~\ref{tab:migrationFunctions} zeigen, dass die Datenbankverbindung mittels der 
Erweiterung \textit{mysql} der größte Schwerpunkt der Migration darstellt. Ähnlich große Verbreitung haben implizite 
Konstruktoren. Die Fehlerklasse \textit{Arraywert per Referenz} wird von \textit{php7mar} zwar ausgewiesen, steht jedoch 
nicht in Zusammenhang mit der Migration zu \ac{PHP} 7.
\begin{table}
    \centering
    \caption{Anteil zu migrierender Codeteile an der gesamten Codebasis}
    \label{tab:migrationPercentage}
    \begin{tabular}{llll}
                        & Gesamt & Betroffen & Anteil   \\
    \textbf{Dateien}    & 5732   & 690       & 12,04\%  \\
    \textbf{Codezeilen} & 596198 & 1431      & 0,24\%  
    \end{tabular}
\end{table}

\begin{table}
    \centering
    \caption{Vorkommen zu migrierender Funktionen in der Codebasis}
    \label{tab:migrationFunctions}
    \begin{tabular}{llll}
                                            & Anzahl fehlerhafter Dateien \\
    \textbf{mysql}                          & 722   \\
    \textbf{Implizite Konstrukoren}         & 453 \\
    \textbf{list}                           & 30 \\
    \textbf{foreach per Referenz}           & 21 \\
    \textbf{Magic Quotes}                   & 20 \\
    \textbf{Indirekte Variablenzugriffe}    & 20 \\
    \textbf{Konstruktoraufruf per Referenz} & 6 \\
    \textbf{preg\_replace mit Option /e}    & 5 \\
    \textbf{Doppelte Funktionsparameter}    & 3 \\    
    \textbf{Arraywert per Referenz}             & 1 \\
    \end{tabular}
\end{table}

\section{Entwicklung von Werkzeugen zur Durchführung der Migration}
Für die Durchführung der Migration, im speziellen die Anpassung der Software an die neue Umgebung, wurde ein Docker-Container 
erstellt, welcher eine einheitliche Konfiguration der Auführungsumgebung bereitstellt. Dazu wurde auf zwei vorgefertigte 
Images zurückgegriffen, eine \ac{PHP}-Installation in Version 7.2 und eine \textit{MySQL}-Datenbank um eine lokale Kopie 
der Datenbank für Tests bereitzustellen. Zudem wurde zur Unterstützung der Entwicklung die Erweiterung \textit{Xdebug}\footnote{Xdebug, \url{https://xdebug.org/}} 
installiert und konfiguriert, wodurch Entwicklern auch bei einer serverseitigen Ausführung des Codes eine Schnittstelle 
für den Debugger der Entwicklungsumgebung bereitgestellt wird.

\section{Entwicklung der an die neue Umgebung angepassten Software}

    \subsection{Ersetzen der Erweiterung mysql}

    \subsection{Ersetzen impliziter Konstruktoren}
    Die Ersetzung impliziter Konstruktoren~\ref{php5implicit} im Quellcode stellt sich im Grunde trivial dar, da der 
    Funktionsname des Konstruktors direkt durch das Schlüsselwort \textit{\_\_construct} ersetzt werden kann. 
    Dadurch bleiben alle Konstruktoraufrufe durch \textit{new} funktionsfähig. Allerdings 
    ist dabei zu beachten, dass die betreffende Konstruktorfunktion auch direkt über ihren Namen aufgerufen werden kann.
    Deshalb ist es ratsam, die ursprüngliche Konstruktorfunktion unverändert zu lassen und diese über den neuen 
    Konstruktor aufzurufen. Funktionsparameter müssen entsprechend vom neuen Konstruktor an die Funktion übergeben werden. 
    Das Beispiel~\ref{lst:php7implicit} zeigt diesen Workaround, der zu maximal möglicher Kompatibilität führt, 
    ohne jeden Aufruf der Funktion zu überprüfen. Dabei 
    wurde der ursprüngliche Konstruktor unverändert belassen, die Funktion \textit{\_\_construct} wurde hinzugefügt. Diese 
    nimmt den Parameter \textit{\$order\_id} entgegen und ruft die Funktion \textit{order} mit diesem auf. Dadurch 
    bleibt \textit{order} innerhalb der Klasse aufrufbar.

    \begin{lstlisting}[language=php, caption={Beispiel der Ersetzung impliziter Konstruktoren}, label={lst:php7implicit}]
        <?php
        class order {
            function __construct($order_id) {
                $this->$order($order_id);
            }

            function order($order_id) {
                print($order_id);
            }
        }
        ?>
    \end{lstlisting}

    \subsection{Überprüfung des Einsatzes von list}

    \subsection{Entfernen aller Aufrufe von Magic Quotes}

    \subsection{Ersetzen von Konstruktoraufrufen per Referenz}

    \subsection{Ersetzen von preg\_replace mit Option /e}
    Die gleiche Funktionsweise wie \textit{preg\_replace} mit der Option \textit{/e} bietet \ac{PHP} seit der Version 4 mit der 
    Funktion \textit{preg\_replace\_callback}. Diese Funktion erwartet neben dem regulären Ausdruck und dem zu prüfenden 
    String eine Callback-Funktion, welche die Treffer des regulären Ausdrucks verarbeitet und eine Ersetzung zurückgibt.
    Beispiel~\ref{lst:php7preg_rep} zeigt, wie \textit{preg\_replace\_callback} die Funktion \textit{preg\_replace} unter 
    Verwendung einer anonymen Funktion ersetzen kann. Die erste Zeile zeigt die ursprüngliche Variante, darunter ist die 
    Ersetzung zu sehen. Diese definiert eine anonyme Funktion, die wiederrum die Funktion \textit{removeEvilAttributes} aufruft. 
    Diese Verkettung geschieht, um maximale Kompatibilität zu gewährleisten, da \textit{removeEvilAttributes} einen String 
    als Argument erwartet, beim direkten Aufruf über \textit{preg\_relace\_callback} ein Array übergeben bekommen würde. Um 
    die Funktion nicht ändern zu müssen und somit möglicherweise Inkompatibilitäten zu schaffen, wurde dieser Workaround 
    gewählt.

    \begin{lstlisting}[language=php, caption={Beispiel der Nutzung von preg\_replace\_callback}, label={lst:php7preg_rep}]
        <?php
        preg_replace('/<(.*?)>/ie', "'<'.removeEvilAttributes('\\1').'>'", $source);
        
        preg_replace_callback(
            '/<(.*?)>/i', 
            function($matches){
                return '<'.removeEvilAttributes($matches[1]).'>';
            },
            $source);
        ?>
    \end{lstlisting}

    \subsection{Entfernen doppelter Funktionsparameter}

    \subsection{}

\section{Durchführung der Migration}

\section{Verifikation der Migration}

\section{Support der alten Umgebung}