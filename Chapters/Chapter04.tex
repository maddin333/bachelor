%************************************************
\chapter{Migration des Tickets75 Onlineshops}\label{ch:migration} 
%************************************************

Dieses Kapitel beschreibt die Migration des Onlineshops der Firma \textit{tickets75} von \ac{PHP} 5.6 zu \ac{PHP} 7.3, sowie die 
genutzten Werkzeuge und Techniken. Die Codebasis 
für das Projekt umfasst 5732 einzelne \ac{PHP}-Dateien, bestehend aus 596.198 Zeilen Code. Diese Menge verdeutlicht, dass 
entsprechende Werkzeuge zur Automatisierung nötig sind, da kein Überblick über die Gesamtheit der Codebasis bestehen kann.

\section{Erstellen der Anforderungsanalyse}\label{analyze}
Die Anforderungsanalyse ist der wichtigste Schritt zur Vorbereitung der Migration. Durch sie kann eine Abschätzung getroffen werden, 
wie schwer und welche Teile des Codes von der Migration betroffen sind. Dies ist vor allem wichtig, hinsichtlich der 
Wirtschaftlichkeit einer Migration. Ist der Anteil zu migrierenden Codes zu hoch, kann es sinnvoll sein, eine Software 
komlett neu zu schreiben. Im vorliegenden Fall wurde die Analyse mit dem Programm \textit{php7mar} durchgeführt.
Die Ergebnisse, dargestellt in Tabelle~\ref{tab:migrationPercentage}, zeigen, dass zwar über 10\% der Dateien unter \ac{PHP} 7 
Fehler enthalten, gemessen an den betroffenen Codezeilen aber nur ein kleiner Teil (0,24\%) des Codes migiriert werden muss.
Bei dieser Zahl ist zu beachten, dass Zeilen mit mehreren zu migrierenden Funktionen nur ein einziges Mal gezählt werden. %TODO
Eine granularere Analyse erlaubt das angehängte Python-Skript, welches die tatsächlich genutzten Funktionen aus dem Report 
herausfiltert. Die Ergebnisse in Tabelle~\ref{tab:migrationFunctions} zeigen, dass die Datenbankverbindung mittels der 
Erweiterung \textit{mysql} den größten Schwerpunkt der Migration darstellt. Ähnlich große Verbreitung haben implizite 
Konstruktoren. Die Fehlerklassen \textit{Arraywert per Referenz} sowie \textit{foreach per Referenz} wird von \textit{php7mar} zwar ausgewiesen, stehen jedoch 
nur bedingt in Zusammenhang mit der Migration zu \ac{PHP} 7. Diese Konstrukte sind keine Fehler, werden aufgrund ihrer Implikationen (beispielsweise des empfohlenen 
Zurücksetzen des Arrayzeigers nach der Benutzung von foreach mit Referenz) von einigen Entwicklern nicht zur Benutzung empfohlen\footnote{Johannes Schlüter, 'References and foreach', \url{https://schlueters.de/blog/archives/141-references-and-foreach.html} (besucht am 30.10.2019)}.
\begin{table}
    \centering
    \caption{Anteil zu migrierender Codeteile an der gesamten Codebasis}
    \label{tab:migrationPercentage}
    \begin{tabular}{llll}
                        & Gesamt & Betroffen & Anteil   \\
    \textbf{Dateien}    & 5732   & 690       & 12,04\%  \\
    \textbf{Codezeilen} & 596198 & 1431      & 0,24\%  
    \end{tabular}
\end{table}

\begin{table}
    \centering
    \caption{Vorkommen zu migrierender Funktionen in der Codebasis}
    \label{tab:migrationFunctions}
    \begin{tabular}{llll}
                                            & Anzahl betroffener Codezeilen \\
    \textbf{mysql}                          & 722 \\
    \textbf{Implizite Konstrukoren}         & 453 \\
    \textbf{funcGetArg}                     & 86 \\
    \textbf{foreach per Referenz}           & 21 \\
    \textbf{Magic Quotes}                   & 20 \\
    \textbf{Indirekte Variablenzugriffe}    & 20 \\
    \textbf{Konstruktoraufruf per Referenz} & 6 \\
    \textbf{preg\_replace mit Option /e}    & 5 \\
    \textbf{Doppelte Funktionsparameter}    & 3 \\
    \textbf{Arraywert per Referenz}         & 1 \\
    \end{tabular}
\end{table}

\section{Entwicklung von Werkzeugen zur Durchführung der Migration}\label{migrationtools}
Für die Durchführung der Migration, im speziellen die Anpassung der Software an die neue Umgebung, wurde ein Docker-Container 
erstellt, welcher eine einheitliche Konfiguration der Auführungsumgebung bereitstellt. Dazu wurde auf zwei vorgefertigte 
Images zurückgegriffen, eine \ac{PHP}-Installation in Version 7.2 und eine \textit{MySQL}-Datenbank um eine lokale Kopie 
der Datenbank für Tests bereitzustellen. Zudem wurde zur Unterstützung der Entwicklung die Erweiterung \textit{Xdebug}\footnote{Xdebug, \url{https://xdebug.org/}} 
installiert und konfiguriert, wodurch Entwicklern auch bei einer serverseitigen Ausführung des Codes eine Schnittstelle 
für den Debugger der Entwicklungsumgebung bereitgestellt wird.

\section{Entwicklung der an die neue Umgebung angepassten Software}\label{develop}

    \subsection{Ersetzen der Erweiterung mysql}\label{mysql}
    Da die Erweiterung \textit{mysql} für Datenbankverbindungen schon seit \ac{PHP} 5.5 veraltet ist, stehen Entwicklern seitdem auch Alternativen zur Verfügung. Mit \textit{mysqli} und 
    \textit{PDO\_MySQL} werden zwei \acp{API} bereitgestellt, deren Funktionalitäten weit über die der alten Erweiterung hinausgehen. Die Entscheidung zwischen 
    beiden Möglichkeiten fiel aufgrund zweier Faktoren auf \textit{mysqli}: Zum einen die syntaktische Ähnlichkeit zwischen \textit{mysql} und \textit{mysqli}, 
    da bei vielen Funktionen nur der \textit{mysql}-Präfix geändert, sowie eine explizite Datenbankverbindung angegeben werden muss. Zum anderen bietet \textit{mysqli} 
    sowohl ein objektorientiertes, als auh ein prozedurales Interface, \textit{PDO\_MySQL} ist dagegen nur objektorientiert nutzbar. Da \textit{mysql} nur 
    prozedural genutzt werden konnte, bietet es sich an, dies voerst beizubehalten und bei Bedarf auf einen objektorientierten Ansatz zu wechseln.
    In einigen Teilen der Software wird für die Datenbankfunktionen eine Fassadenklasse nach der in Kapitel~\ref{facade} 
    dargestellten Methode verwendet. Hier muss neben des Austauschs der einzelnen \textit{mysql}-Funktionen noch eine 
    Variable eingeführt werden, welche die aktuelle Datenbankverbindung hält.
    Da diese Fassadenklasse nicht in allen Teilen der Software zum Einsatz kommt, muss die Datenbankverbindung hier vorerst 
    über eine globale Variable gesetzt werden. In einem späteren Refactoring sollten dies durch den Einsatz der Fassadenklasse 
    ersetzt werden. Beispiel~\ref{lst:php7mysqlilow} zeigt den Austausch der veralteten \textit{mysql}-Funktionen \textit{mysql\_query} und \textit{mysql\_error} 
    durch die neuen Funktionen der \textit{mysqli}-Erweiterung. Dabei ist die explizite Angabe der Datenbankverbindung in beiden Funktionen besonders hervorzuheben. 
    Eine Schwierigkeit ergibt sich bei \textit{mysql}-Funktionen, welche kein entsprechendes Pendant in der neuen Erweiterung besitzen. Dies wird in Beispiel~\ref{lst:php7mysqlihard} 
    gezeigt. \textit{mysql\_result} liefert ein einzelnes Feld aus einer angegebenen Zeile einer Abfrage. Um das selbe Verhalten in \textit{mysqli} zu erzeugen muss 
    der Ergebniszeiger mit \textit{mysqli\_data\_seek} auf die entsprechende Zeile gerichtet werden, welche anschließend Feld für Feld durchlaufen wird um das entsprechende 
    Ergebnis zu liefern.

    \begin{lstlisting}[language=php, caption={Beispiel der Ersetzung von \textit{mysql} durch \textit{mysqli}}, label={lst:php7mysqlilow}]
        <?php
        //veralteter Aufruf von mysql_query & mysql_error
        $gm_query_result = mysql_query($gm_query[$i]);
        if(!$gm_query_result) $gm_query_result_output .= GM_SQL_ERROR . mysql_error() . '<br />';

        //Ersatz durch mysqli_query & mysqli_error
        $gm_query_result = mysqli_query($db_link, $gm_query[$i]);
		if(!$gm_query_result) $gm_query_result_output .= GM_SQL_ERROR . mysqli_error($db_link) . '<br />';
        ?>
    \end{lstlisting}

    \begin{lstlisting}[language=php, caption={Beispiel der Ersetzung von \textit{mysql\_result} durch \textit{mysqli}}, label={lst:php7mysqlihard}]
        <?php
        //veraltete Funktion mysql_result
        mysql_result(xsb_db_query("SELECT MIN(_EBAY_START_TIME) as MinTime FROM xtb_auctions"),0,'MinTime');

        //Ersatz durch mysqli_data_seek
        $result = xsb_db_query("SELECT MIN(_EBAY_START_TIME) as MinTime FROM xtb_auctions");
        $field = 'MinTime';
        mysqli_data_seek($result, 0);
        if(!empty($field)) {
            while($fieldInfo = mysqli_fetch_field($result)) {
                if( $field == $fieldInfo->name ) {
                    $row = mysqli_fetch_assoc($result);
                    $fetch =  $row[$field];
                }
            }
        } else {
            $row = mysqli_fetch_array($result);
            $fetch = $row[0];
        }
        ?>
    \end{lstlisting}

    \subsection{Ersetzen impliziter Konstruktoren}\label{construct}
    Die Ersetzung impliziter Konstruktoren (vgl. Kapitel~\ref{php5implicit} im Quellcode stellt sich im Grunde trivial dar, da der 
    Funktionsname des Konstruktors direkt durch das Schlüsselwort \textit{\_\_construct} ersetzt werden kann. 
    Dadurch bleiben alle Konstruktoraufrufe durch \textit{new} funktionsfähig. Allerdings 
    ist dabei zu beachten, dass die betreffende Konstruktorfunktion auch direkt über ihren Namen aufgerufen werden kann.
    Deshalb ist es ratsam, die ursprüngliche Konstruktorfunktion unverändert zu lassen und diese über den neuen 
    Konstruktor aufzurufen. Funktionsparameter müssen entsprechend vom neuen Konstruktor an die Funktion übergeben werden. 
    Das Beispiel~\ref{lst:php7implicit} zeigt diesen Workaround, der zu maximal möglicher Kompatibilität führt, 
    ohne jeden Aufruf der Funktion zu überprüfen. Dabei 
    wurde der ursprüngliche Konstruktor unverändert belassen, die Funktion \textit{\_\_construct} wurde hinzugefügt. Diese 
    nimmt den Parameter \textit{\$order\_id} entgegen und ruft die Funktion \textit{order} mit diesem auf. Dadurch 
    bleibt \textit{order} innerhalb der Klasse aufrufbar.

    \begin{lstlisting}[language=php, caption={Beispiel der Ersetzung impliziter Konstruktoren}, label={lst:php7implicit}]
        <?php
        class order {
            function __construct($order_id) {
                $this->$order($order_id);
            }

            function order($order_id) {
                print($order_id);
            }
        }
        ?>
    \end{lstlisting}

    %   \subsection{}
    
    \subsection{Entfernen aller Aufrufe von Magic Quotes}
    Wie in Kapitel~\ref{php5magicquotes} ausgeführt, haben Aufrufe der Funktionen \textit{magic\_quotes\_runtime} sowie \textit{set\_magic\_quotes\_runtime} seit \ac{PHP} 5.4 
    keinen Effekt mehr. Daher können diese bei einer Migration von \ac{PHP} 5.6 gefahrlos entfernt werden. Trotzdem ist darauf zu achten, dass Inhalte aus 
    externen Quellen mit angemessenen Funktionen maskiert werden, um Sicherheitsrisiken zu vermeiden. Die Erweiterung \textit{mysqli} bietet dafür beispielsweise 
    die Funktion \textit{mysqli\_real\_escape\_string}, welche Sonderzeichen unter Berücksichtigung des eingestellten Zeichensatzes maskiert.
    
    \subsection{Korrektur indirekter Variablenzugriffe}\label{indirect}
    Die in Kapitel~\ref{indirectVar} gezeigte Einführung einer strikten Evaluierung von Variablen von links nach rechts führt dazu, 
    dass indirekte Variablenzugriffe aus \acs{PHP} 5 in \acs{PHP} 7 falsch evaluiert werden. Die korrekte Evaluierung kann durch 
    die Einführung geschweifter Klammern erzwungen werden. Beispiel~\ref{lst:php7newIndirect} zeigt die vorgenommene Änderung am 
    indirekten Zugriff auf die Session-Variable. Dabei ist anzumerken, dass die ursprüngliche Version nur unter \acs{PHP} 5 
    korrekt ausgeführt wird, die angepasste Version jedoch sowohl unter \acs{PHP} 7 als auch unter \acs{PHP} 5. 
    \begin{lstlisting}[language=php, caption={Anpassung indirekter Variablenzugriffe}, label={lst:php7newIndirect}]
        <?php
        if (isset ($$_SESSION['payment']->form_action_url)) {
            $form_action_url = $$_SESSION['payment']->form_action_url;
        }
        
        //wird zu:

        if (isset (${$_SESSION}['payment']->form_action_url)) {
            $form_action_url = ${$_SESSION}['payment']->form_action_url;
        }       
        ?>
    \end{lstlisting}
    
    \subsection{Ersetzen von Konstruktoraufrufen per Referenz}
    Die Ersetzung von Konstruktoraufrufen, die per Referenz zugewiesen werden ist trivial. Es genügt, bei allen Vorkommen die Referenz zu 
    entfernen und durch einen Zuweisungsoperator zu ersetzen, da der Operator \textit{new} seit \ac{PHP} 5 automatisch eine Referenz zurückgibt. 
    Die Ersetzung wird im Beispiel~\ref{lst:php7newReference} deutlich. Der auf den Konstruktoraufruf folgende Code kann unverändert mit dem Objekt arbeiten.
    
    \begin{lstlisting}[language=php, caption={Beispiel der Ersetzung von Konstruktoraufrufen per Referenz}, label={lst:php7newReference}]
        <?php
        $rv =& new $class($this); //alter Konstruktoraufruf per Refrenz
        $rv = new $class($this); //neuer Konstruktoraufruf ohne Referenz
        
        if(is_subclass_of($rv, 'IclearBase')) {
            $rv->icError =& $this->getObject('IclearError', true);
            }
        ?>
    \end{lstlisting}
    
    \subsection{Ersetzen von preg\_replace mit Option /e}\label{preg}
    Die gleiche Funktionsweise wie \textit{preg\_replace} mit der Option \textit{/e} bietet \ac{PHP} seit der Version 4 mit der 
    Funktion \textit{preg\_replace\_callback}. Diese Funktion erwartet neben dem regulären Ausdruck und dem zu prüfenden 
    String eine Callback-Funktion, welche die Treffer des regulären Ausdrucks verarbeitet und eine Ersetzung zurückgibt.
    Beispiel~\ref{lst:php7preg_rep} zeigt, wie \textit{preg\_replace\_callback} die Funktion \textit{preg\_replace} unter 
    Verwendung einer anonymen Funktion ersetzen kann. Die erste Zeile zeigt die ursprüngliche Variante, darunter ist die 
    Ersetzung zu sehen. Diese definiert eine anonyme Funktion, die wiederrum die Funktion \textit{removeEvilAttributes} aufruft. 
    Diese Verkettung geschieht, um maximale Kompatibilität zu gewährleisten, da \textit{removeEvilAttributes} einen String 
    als Argument erwartet, beim direkten Aufruf über \textit{preg\_relace\_callback} ein Array übergeben bekommen würde. Um 
    die Funktion nicht ändern zu müssen und somit möglicherweise Inkompatibilitäten zu schaffen, wurde dieser Workaround 
    gewählt.
        
    \begin{lstlisting}[language=php, caption={Beispiel der Nutzung von preg\_replace\_callback}, label={lst:php7preg_rep}]
        <?php
        preg_replace('/<(.*?)>/ie', "'<'.removeEvilAttributes('\\1').'>'", $source);
        
        preg_replace_callback(
            '/<(.*?)>/i', 
        function($matches){
            return '<'.removeEvilAttributes($matches[1]).'>';
            },
            $source);
        ?>
    \end{lstlisting}
            
    \subsection{Entfernen doppelter Funktionsparameter}
    Wie in Kapitel~\ref{php5double} beschrieben, führt die Angabe mehrerer Funktionsparameter mit dem selben Namen unter \ac{PHP} 7 zu einem Fehler.
    Um diesen zu korrigieren reicht es nicht aus, die überflüssigen Parameter zu löschen. Vielmehr müssen alle Aufrufe der Funktion überprüft werden, 
    um entscheiden zu können, welches Vorgehen sinnvoll ist. Sind die Parameter wie in Aufruf 1 im Beispiel~\ref{lst:php7double} mit der gleichen Variable 
    belegt, genügt das Löschen eines Parameters in der Funktion und im Aufruf. Sind die Parameter jedoch unterschiedlich belegt, darf, um die Funktionalität 
    zu erhalten, nur der letzte Parameter bestehen bleiben, da dessen Wert innerhalb der Funktion zum Tragen kommt.
    
    \begin{lstlisting}[language=php, caption={Beispiel der Nutzung von preg\_replace\_callback}, label={lst:php7double}]
        <?php
        function foo($bar, $bar) {
            echo($bar);
            }
            
            //Aufruf 1
            foo($x,$x);
            //Aufruf 2
            foo($x,$y);
            ?>
        \end{lstlisting}
                
\section{Durchführung der Migration}
Für die Ausführung der im vorangegangenen Kapitel entwickelten Software wird eine neue Ausführungsumgebung mit \ac{PHP} 7 benötigt. Dazu wurden zwei neue 
Server angemietet, um die alte Infrastruktur nachzubilden und beide Systeme für Tests an der neuen Umgebung parallel zueinander zu betreiben. Auf den neuen 
Servern, die identische Konfigurationen zu den bisher zum Einsatz gekommenen Geräten aufweisen, wurde \ac{PHP} 7.2 als Ausführungsumgebung installiert. 
Als Software für den Webserver wird Apache\footnote{Apache HTTP Server Project, \url{https://httpd.apache.org/}} eingesetzt. Nach erfolgreichen Tests konnte die 
alte Infrastruktur durch die neue ersetzt werden, die migrierte Software also produktiv eingesetzt werden. Durch den vorgeschalteten Load-Balancer konnte jeweils 
ein Server aus dem Produktivbetrieb entfernt werden und durch den neuen ersetz werden. Dadurch konnte die Webseite mit geringer Ausfallzeit migiriert werden und 
im Falle eines Fehlers hätte schnell auf die alte Infrastruktur zurückgegriffen werden können.

\section{Verifikation der Migration}\label{newrelic}
Zur Verifikation der Migration wurden zum einen eigene Tests durchgeführt, vor allem verschiedene Testkäufe um die Funktion für Kunden des Onlineshops zu verifizieren, 
sowie administrative Tätigkeiten wie die Anpassung der Startseite und das Hinzufügen von Produkten zum Katalog. Zum Anderen wurde die Funktion der Webseite durch das 
Monitoring-Tool \textit{New Relic}\footnote{New Relic, \url{https://newrelic.de}} überwacht. Dieses Werkzeug überwacht die Performance der Applikation und meldet Fehler 
in der Ausführung. Durch diese Überwachung wäre es möglich gewesen, möglichst schnell auf etwaige Fehler zu reagieren und geeignete 
Gegenmaßnahmen zu ergreifen, bspw. einen Rollback auf die alte Infrastruktur oder einen Hotfix der neuen Version. 
Auf die Einführung von Unit Tests im Projekt musste leider verzichtet werden, da bisher keine Infrastruktur dafür vorhanden ist 
und eine nachträgliche Einführung in einem zeitlich angemessenen Rahmen nicht möglich war.

\section{Support der alten Umgebung}
Diese Anforderung im Standard \textbf{ISO/IEC 14764} betrifft vor allem Software, die an Kunden ausgeliefert wird und dort möglicherweise auch nach der Migration 
zum Einsatz kommt und weiter gepflegt werden soll. Für den vorliegenden Fall ist dies nur relevant, falls signifikante Fehler in der migrierten Software auftreten, 
die einen weiteren Einsatz unmöglich machen und den erneuten Einsatz der alten Software und Ausführungsumgebung erfordern.
Zu diesem Zweck wurde für eine Übergangszeit von einem Monat die alte Infrastruktur weiter in Betrieb gehalten, bevor diese abgeschaltet wurde. Der ursprüngliche 
Code, der nicht unter \ac{PHP} 7 lauffähig ist, steht mitsamt einer lauffähigen Ausführungsumgebung, wie in Kapitel~\ref{VCS} beschrieben, weiter über die 
Versionsverwaltung zur Verfügung.

\section{Zusammenfassung}
Dieses Kapitel beschreibt die Migration der Software nach dem Standard \textbf{ISO/IEC 14764}. \\
Im ersten Abschnitt wird eine Anforderungsanalyse durchgeführt, die den Umfang der Migration fetlegt. Diese zeigt, dass 
zwar über 10\% der Dateien im vorliegenden Projekt migriert werden müssen, dabei jedoch nur ein kleiner Teil der Codezeilen 
betroffen ist. \\
Der Abschnitt~\ref{migrationtools} geht auf die Entwicklung von Containern mittels der Software \textit{Docker} ein. \\
Die eigentliche Entwicklung der angepassten Software wird im Abschnitt~\ref{develop} beschrieben. Dabei wird im einzelnen 
auf die veralteten Funktionen und deren Überführung zu \acs{PHP} 7 eingegangen. \\
Die letzten Abschnitte zeigen den Ablauf der Migration, wie diese verifiziert wird und wie die geforderte Unterstützung 
der alten Umgebung realisiert wurde.