%************************************************
\chapter{Migration des Tickets75 Onlineshops}\label{ch:migration} 
%************************************************

Dieses Kapitel beschreibt die Migration des Onlineshops der Firma \textit{tickets75} von \ac{PHP} 5.6 zu \ac{PHP} 7.3, sowie die 
genutzten Werkzeuge und Techniken. Die Codebasis 
für das Projekt umfasst 5732 einzelne \ac{PHP}-Dateien, bestehend aus 596.198 Zeilen Code. Diese Menge verdeutlicht, dass 
entsprechende Werkzeuge zur Automatisierung nötig sind, da kein Überblick über die Gesamtheit der Codebasis bestehen kann.

\section{Anforderungsanalyse}
Die Anforderungsanalyse ist der wichtigste Schritt zur Vorbereitung der Migration. Durch sie kann eine Abschätzung getroffen werden, 
wie schwer und welche Teile des Codes von der Migration betroffen sind. Dies ist vor allem wichtig, hinsichtlich der 
Wirtschaftlichkeit einer Migration. Im vorliegenden Fall wurde die Analyse mit dem Programm \textit{php7mar} durchgeführt.
Die Ergebnisse, dargestellt in \ref{tab:migrationPercentage}, zeigen, dass zwar über 10\% der Dateien unter \ac{PHP} 7 
Fehler enthalten, gemessen an den betroffenen Codezeilen aber nur ein kleiner Teil (unter 0,5\%) des Codes migiriert werden muss.
 
\begin{table}
    \centering
    \caption{Anteil zu migrierender Codeteile an der gesamten Codebasis}
    \label{tab:migrationPercentage}
    \begin{tabular}{llll}
                        & Gesamt & Betroffen & Anteil   \\
    \textbf{Dateien}    & 5732   & 690       & 12,04\%  \\
    \textbf{Codezeilen} & 596198 & 1431      & 0,24\%  
    \end{tabular}
\end{table}
