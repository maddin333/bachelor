%************************************************
\chapter{Grundlagen}\label{ch:einführung}
%************************************************
Dieser Bericht ist enthält eine Zusammenfassung meiner Tätigkeiten im Rahmen des Pflichtmoduls \textit{Praxisprojekt} im Bachelorstudiengang Medieninformatik an der HTWK Leipzig.\\
Das Praxisprojekt fand zwischen dem 17. März 2019 und dem 07. Juni 2019 in Kooperation mit dem Unternehmen
\begin{center}
	TICKETS75\\
	Büsumer Straße 40–44\\
	24768 Rendsburg
\end{center}
am Standort Leipzig statt.

\section{Firmenportrait}
\textit{Tickets75} ist eine 2006 gegründete Kartenagentur, die sich auf die Vermittlung von begehrten Tickets für Konzerte, Sportveranstaltungen und Unterhaltungsshows spezialisiert hat.
Das Unternehmen beschäftigt etwa 30 Mitarbeiter an zwei Standorten. Am Standort Leipzig befinden sich die Abteilungen \textit{Produktmanagement \& Verkauf}, sowie \textit{IT \& Online-Marketing}. 
In Rendsburg befinden sich die Versandabteilung sowie der offiziele Firmenhauptsitz. Die Entwicklung des Webshops wird zusätzlich durch zwei externe Entwickler
unterstützt.

\section{Entwicklungsumgebung \& verwendete Technologien}
Der Webshop des Unternehmens basiert auf dem Shopsystem \textit{Gambio}, einer Weiterentwicklung des Open-Source-Projekts \textit{xt:Commerce}. 
Die eingesetzte Version ist stark an die eigenen Bedürfnisse angepasst und muss deshalb eigenständig weiterentwickelt werden. Basis des Webshops bildet \textit{php}, 
dem eine Datenbank auf Basis von \textit{MySQL} zur Seite steht. Das Templating von Inhaltsseiten wird durch das Framework \textit{Smarty} abgewickelt. 
Die lokale Entwicklung kann auf zweierlei Arten stattfinden. Eine Möglichkeit besteht in der Verwendung des Softwarepakets \textit{XAMPP}, 
welches den Webserver \textit{Apache} und \textit{MySQL} für Windows bereitstellt. Eine andere Möglichkeit besteht in der Verwendung von \textit{Docker}, 
einer Software zur Containervirtualisierung. Der Vorteil dieser Praxis liegt in der vereinheitlichten Umgebung, die unabhängig von individuellen Systemeinstellungen ist. 
Das gesamte Projekt wird in \textit{Bitbucket}, einem System zur Versionsverwaltung, vorgehalten und kann lokal durch \textit{git} gesteuert werden. 
Als Entwicklungsumgebung wird \textit{Visual Studio Code} verwendet, das durch Plugins auf die eigenen Bedürfnisse angepasst werden kann. 
Dazu zählen Erweiterungen zur Autovervollständigung und zum Debuggen mittels \textit{XDebug}.

\section{Organisation}
Die Entwicklung am Webshop ist in zweiwöchige Sprints aufgeteilt. Inhalte des Sprints werden zu Beginn in einem \textit{Planning Meeting} besprochen und festgelegt.
Dabei wird für neu hinzugekommene Tickets auch der Aufwand abgeschätzt. Der Sprint wird durch tägliche Scrum-Meetings unterstützt, in denen jeder Entwickler einen 
kurzen Überblick über seine Arbeit des letzten, sowie des kommenden Tages gibt. Gleichzeitig dient dieses Meeting dazu, etwaige Fragen und offene Probleme zu klären.
Während eines Sprints erfolgen ein bis zwei Releases fertiggestellter Aufgaben, so dass wöchentlich eine aktualisierte Version produktiv eingesetzt werden kann.
Den Abschluss eines Sprints bildet die Retrospektive. Dabei wird von jedem beteiligten Kritik - sowohl positiv, als auch negativ - geäußert und gemeinsam
besprochen, wie aufgetauchten Problemen entgegengewirkt werden kann.