%************************************************
\chapter{Grundlagen}\label{ch:grundlagen}
%************************************************

\section{Softwarewartung nach ISO/IEC 14764}
Die \ac{ISO} ist ein im Jahr 1947 gegründeter Zusammenschluss internationaler Normungskommissionen, mit dem Ziel internationale Standards zu entwickeln und zu etablieren.\cite{international_organization_for_standardization:_about_nodate}
Die Entwicklung von Standards wird von der 1906 gegründeten Schwesterorganisation \ac{IEC} übernommen, oftmals in Zusammenarbeit mit der \ac{ISO}.\cite{international_electrotechnical_commission_iec_nodate}
Aus dieser Zusammenarbeit entstandene Standards tragen die Kürzel beider Organisationen im Namen. Ein solcher Standard ist \textbf{ISO/IEC 14764} mit dem Titel
\textbf{Software Engineering — Software Life Cycle Processes — Maintenance}, der erstmals im Jahr 1999 veröffentlicht wurde.
\textbf{ISO/IEC 14764} normiert den Prozess der Wartung von Software bis zu deren Einstellung.
Darin wird unter Anderem beschrieben, welche Schritte bei der Migration von Software zu befolgen sind, sobald diese an eine neue Umgebung angepasst werden muss.
Folgende Aktionen sind durch den Ausführenden nach \textbf{ISO/IEC 14764} umzusetzen:
\begin{itemize}
    \item Analyse der Anforderungen und Definition der Migration
    \item Entwicklung von Werkzeugen zur Migration
    \item Entwicklung der an die neue Ugebung angepassten Software
    \item Durchführung der Migration
    \item Verifikation der Migration
    \item Support der alten Umgebung
\end{itemize}

\section{PHP}
\textbf{\ac{PHP}} ist eine Skriptsprache, welche seit 1994 entwickelt wird und seit 1995 Open-Source bereitgestellt wird.
Obwohl \textbf{PHP} viele Einsatzzwecke abdeckt, wird es hauptsächlich dazu genutzt, dynamische Websites zu programmieren.

\section{Versionierung}

