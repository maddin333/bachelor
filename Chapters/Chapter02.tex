%*****************************************
\chapter{Änderungen der PHP-API}\label{ch:php7}
%*****************************************
Dieser Abschnitt stellt eine Auswahl der Bedingungen vor, welche die \acs{PHP}-\acs{API} in Version 7 gegenüber Version 5 an lauffähige Software stellt und welche 
neuen Mittel Entwicklern zur Verfügung gestellt werden. Die Änderungen werden in den Kontext der Weiterentwicklung der Programmiersprache gestellt, um Aussagen 
über die Gründe dieser zu treffen und - \textbf{ISO/IEC 14764} entsprechend - die Anforderungen an die Migration festzustellen.

\section{Abwärtsinkompatible Änderungen}
Änderungen in dieser Kategorie führen in älteren Versionen zu Fehlern oder unerwartetem Verhalten und sind in dieser Umgebung somit nicht lauffähig. Durch diese wird ein 
Wechsel der Ausführungsumgebung zwingend vorrausgesetzt.

    \subsection{Interpretation indirekter Variablenzugriffe}\label{indirectVar}
    \acs{PHP} bietet die Möglichkeit des indirekten Zugriffs auf Variablen. Das bedeutet, dass der Wert einer Variablen den Namen einer weiteren Variablen darstellt.
    Bisher war die Syntax durch mehrere Sonderfälle geregelt. Mit \acs{PHP} 7 wird eine strikte Evaluierung eines solchen Audrucks von links nach rechts eingeführt, 
    um die Nutzung dieser zu vereinheitlichen. Wie sich die einzelnen Fälle Unterscheiden ist in Tabelle~\ref{tab:indirect-expressions} aufgeführt.

    \begin{table}[]
        \caption{Vergleich der Evaluation indirekter Variablen zwischen PHP 5 und PHP 7}
        \label{tab:indirect-expressions}
        \begin{tabular}{lll}
        \textbf{Ausdruck}               & \textbf{PHP 5}                    & \textbf{PHP 7}                \\ \hline
        \$\$foo{[}'bar'{]}{[}'baz'{]}   & \$\{\$foo{[}'bar'{]}{[}'baz'{]}\} & (\$\$foo){[}'bar'{]}{[}'baz'{]} \\ \hline
        \$foo->\$bar{[}'baz'{]}         & \$foo->\{\$bar{[}'baz'{]}\}       & (\$foo->\$bar){[}'baz'{]}       \\ \hline
        \$foo->\$bar{[}'baz'{]}()       & \$foo->\{\$bar{[}'baz'{]}\}()     & (\$foo->\$bar){[}'baz'{]}()     \\ \hline
        Foo::\$bar{[}'baz'{]}()         & Foo::\{\$bar{[}'baz'{]}\}()       & (Foo::\$bar){[}'baz'{]}()    
        \end{tabular}
    \end{table}

    \subsection{Switch-Anweisungen mit mehreren default-Blöcken}\label{switch}
    Switch-Anweisungen, welche mehrere default-Blöcke enthalten werden ab sofort als fehlerhafte Syntax erkannt und werfen einen Fehler. Dies war bisher nicht der Fall,
    allerdings wurde bei einer solchen Anweisung nur der letzte default-Block ausgewertet. Dieses Verhalten zeigt sich in Listing~\ref{lst:php5switch}. Der entsprechende
    Codeausschnitt gibt unter \acs{PHP} 5 immer "Evaluated" aus, bei dem Versuch der Auführung unter \acs{PHP} 7 wird ein Fehler geworfen. Damit wird ein Bruch der
    \acs{PHP}-Spezifikation \cite{php_group_php_nodate} behoben.

    \begin{lstlisting}[language=php, caption={Beispiel meherer default-Blöcke in Switch-Anweisungen}, label={lst:php5switch}]
        <?php
        switch (1) {
            default:
                echo("Never evaluated");
                break;
            default:
                echo("Evaluated")
                break;
        }
        ?>
    \end{lstlisting}

    \subsection{Verkehrte Reihenfolge der Variablenzuweisung mit list}
    Die Funktion \textit{list()} ermöglicht die Zuweisung von Variablen als wären diese ein Array. Quellcode der sich auf die bisherige Praxis verlässt, dass \textit{list()}
    den letzten angegebenen Wert zuerst zuweist, kann nun nicht mehr eingesetzt werden, da die Reihenfolge der Zuweisung umgekehrt wurde. 
    Obwohl keine klaren Gründe auszumachen sind, liegt die Vermutung nahe, dass die Änderung Verwirrungen über das Verhalten der Funktion vermindern soll. Listing
    \ref{lst:php5list} würde bei der Ausführung unter \acs{PHP} 5 beispielsweise \glqq3\grqq{} als Ergebnis ausgeben. Dies entspricht nicht der erwartbaren Reihenfolge.

    \begin{lstlisting}[language=php, caption={Beispiel der Verwendung von list()}, label={lst:php5list}]
        <?php
        list($first, $second, $third) = [1,2,3];

        echo($first);
        ?>
    \end{lstlisting}

\section{Veraltete Funktionen}
Als veraltet markierte Funktionen sind in der neuen Umgebung zwar noch unterstützt, sollten aber nach Möglichkeit nicht mehr eingesetzt und schnellstmöglich durch geeignete 
Funktionen ersetzt werden, da sie möglicherweise in zukünftigen Versionen entfernt oder verändert werden. Werden diese Funktionen trotzdem eingesetzt, wird eine Warnung 
ausgegeben, die Programmierer darauf hinweisen soll, dass die Verwendung der Funktion möglicherweise gefährlich sein kann. Die Lauffähigkeit des Programms wird bis zur 
abschließenden Entfernung der Funktion jedoch nicht beeinflusst. \cite{oracle_how_2004}
    \subsection{Implizite Benennung von Konstruktoren}
    Mit der Einführung der objektorientierten Programmierung in PHP 4 wurde festgelegt, dass Funktionen mit dem selben Namen wie die umschließende Klasse implizit als 
    Konstruktor der Klasse erkannt werden. Ein Beispiel zur Implementierung eines Konstruktors nach diesem Prinzip ist in Listing~\ref{lst:php4construct} dargestellt.
    PHP 7 unterstützt diese Notation zwar noch, allerdings wird die, in PHP 5 eingeführte, explizite Benennung mit dem Schlüsselwort \textit{\_\_construct} (siehe 
    Listing~\ref{lst:php5construct}) bevorzugt. Hierdurch soll die Verwirrung darum, wann eine Funktion einen Konstruktor darstellt aufgehoben werden. \cite{morrison_php:_2014}
    \begin{lstlisting}[language=php, caption={Beispiel eines impliziten Konstruktors}, label={lst:php4construct}]
        <?php
        class foo {
            function foo($a) {
                echo("Created instance of class 'foo'");
            }
        }
        ?>
    \end{lstlisting}

    \begin{lstlisting}[language=php, caption={Beispiel eines expliziten Konstruktors}, label={lst:php5construct}]
        <?php
        class foo {
            function __construct($a) {
                echo("Created instance of class 'foo'");
            }
        }
        ?>
    \end{lstlisting}
    
    \subsection{Statische Aufrufe nicht-statischer Funktionen}
    Mit dem Schlüsselwort \textit{static} versehene Funktionen einer Klasse erlauben das Benutzen der Funktion, ohne die 
    Instantiierung der Klasse selber. Damit steht die entsprechende Funktion nicht im Kontext eines Objekts, sondern 
    im Kontext der entsprechenden Klasse. Im Gegensatz zu anderen objektorientierten Programmiersprachen (bspw. Java) war es 
    in PHP bisher möglich, auch nicht-statische Methoden ohne eine Instantiierung zu verwenden. Diese Möglichkeit wurde mit
    PHP 7 für veraltet erklärt und sollte nicht mehr genutzt werden. Dadurch werden Programmierfehler verhindert, da der Kontext, in dem eine Funktion ausgeführt wird nun
    Eindeutig ist. Das Beispiel~\ref{lst:php7static} wird eine Warnung ausgeben, dass eine nicht-statische Methode statisch aufgerufen wird.
    \begin{lstlisting}[language=php, caption={Beispiel eines statischen Aufrufs einer nicht-satischen Funktion in PHP 7}, label={lst:php7static}]
        <?php
        class foo {
            function bar() {
                echo("'bar' is not a static function");
            }
        }

        foo::bar();
        ?>
    \end{lstlisting}

\section{Geänderte Funktionen}
In diese Gruppe fallen Funktionen, deren Benutzung und/oder Verhalten geändert wurden, allerdings nicht vollständig veraltet sind. Dies bedeutet zum Beipiel, dass 
einzelne Funktionsparameter entfernt wurden oder andere Datentypen zurückgegeben werden.
    \subsection{preg\_replace} \label{preg_replace}
    Die Funktion \textit{preg\_replace()} ersetzt Teile einer Zeichenkette nach einem, als regulärem Ausdruck angegebenen, Muster. Mit \textit{\acs{PCRE}-Modifikatoren} 
    kann die Verhaltensweise des regulären Ausdrucks gesteuert werden. In \acs{PHP} 7 wurde der Modifikator \textit{/e} entfernt, mit dem die Zeichenkette durch das Ergebnis
    einer Funktion ersetzt wird. Ein Beipiel ist die Umwandlung aller kleingeschriebenen Zeichen eines Strings in Großbuchstaben, dargestellt in Listing~\ref{lst:php5preg_rep}. Die Verwendung
    des Modifikators wird aufgrund der Maskierungsregeln für bestimmte Zeichen als sehr kompliziert beschrieben. Gleichzeitig stellt die einfache Art der Evaluierung
    des Ergebnisses keine Schutzmechanismen zur Verfügung, wodurch Sicherheitslücken entstehen können, sobald es einem Angreifer gelingt, ausfühbaren Code in diese
    Funktion einzuschleusen. 

    \begin{lstlisting}[language=php, caption={Beispiel der Nutzung von preg\_replace mit dem Modifikator /e}, label={lst:php5preg_rep}]
        <?php
        $uppercase = preg_replace(
            "/([a-z]*)/e",
            "strtoupper($1)",
            $mixedCase
        );
        ?>
    \end{lstlisting}

    \subsection{setlocale}
    Die Funktion \textit{setlocale()} dient dazu, regionale Eigenheiten abzubilden. Dazu gehören zum Beispiel unterschiedliche Datumsformate oder die Formatierung von 
    Zahlen (bspw. Trennzeichen für Dezimalzahlen). Für die Einstellung einer Region können Kategorien angegeben werden, auf die sich die Änderung auswirken soll. 
    Ab Version 7 ist es nicht mehr möglich, die Kategorie als Zeichenkette anzugeben. Für diese Änderung ist kein Grund angegeben, allerdings liegt die Vermutung
    nahe, dass sich dadurch die Prüfung der Kategorie innerhalb der Funktion vereinfachen lässt, da \acs{PHP} verschiedene benannte Konstanten zur Anwendung zur Verfügung 
    stellt. Dies lässt sich auch durch die Historie der betreffenden Funktion im Quellcode belegen, durch die ersichtlich wird, dass ein großer Teil der Überprüfung
    der Funktionsparameter entfernt wurde. \cite{nikic_remove_2014}

\section{Neue Funktionen}
    \subsection{Anonyme Klassen}
    Mit dem Hinzufügen von anonymen Klassen implementiert \acs{PHP} ein Konzept, das bereits aus anderen Objektorientierten Sprachen, beispielsweise Java 
    \cite{oracle_anonymous_nodate}, bekannt ist. Diese können benutzt werden, um gleichzeitig mit der Definition eine einmalig genutzte Klasse zu instanziieren, ohne eigens 
    dafür eine neue lokale Klasse erstellen zu müssen., wie in Listing~\ref{lst:php7anon_class} dargestellt wird.

    \begin{lstlisting}[language=php, caption={Beispiel der Nutzung anonymer Klassen}, label={lst:php7anon_class}]
        <?php
        $foo = new class {
            public function bar() {
                echo "Hello World";
            }
        };

        $foo->bar();
        ?>
    \end{lstlisting}

    \subsection{preg\_replace\_callback\_array()}
    Ähnlich wie die im Abschnitt \ref{preg_replace} beschriebene Funktion \textit{preg\_replace()} mit dem Modifikator \textit{/e}, ersetzt 
    \textit{preg\_replace\_callback\_array()} Zeichenketten anhand eines Musters und einer Ersetzungsfuntion. Im eingeführten \textit{preg\_replace\_callback\_array()} 
    kann nun ein assiozatives Array angegeben werden, das mehrere Muster und ihre entsprechenden Callback-Funktionen enthält. Durch die Nutzung verschiedener 
    Ersetzungsfunktionen kann auf die Nutzung einer einzelnen, stark verzweigten Ersetzungsfuntion verzichtet werden. Dadurch wird entsprechender Quellcode lesbarer 
    und besser wartbar (vgl. \cite[S. 34f]{martin_clean_2012}).

    \subsection{Typdeklaration für Rückgabewerte} %TODO: Referenz PHP Typen
    Als schwach typisierte Sprache bot \acs{PHP} bisher keine Möglichkeit der Deklaration von Typen für Rückgabewerte von Funktionen. Dies kann nun durch Angabe des
    Typs zwischen Funktionsdeklaration und dem Code der Funktion geschehen, wie in Listing~\ref{lst:php7returnvalues}. Dadurch sollen unter anderem ungewollte Rückgabewerte verhindert werden, als auch die 
    automatisierte Dokumentation von Funktionen vereinfacht werden. \cite{morrison_php:_2014-1} 

    \begin{lstlisting}[language=php, caption={Typdeklaration für Rückgabewerte}, label={lst:php7returnvalues}]
        <?php
        public function foo(): int {
            return 42;
        }
        ?>
    \end{lstlisting}

\section{Entfernte Erweiterungen}
Einige Funktionalitäten von \acs{PHP} sind nicht in die Sprache selbst eingebaut, sondern werden durch externe Erweiterungen eingebunden, die jedoch standardmäßig
mit \acs{PHP} ausgeliefert werden. Diese stehen somit nicht unter der Verwaltung der \acs{PHP}-Entwickler und werden unabhängig weiterentwickelt.
    \subsection{mysql}
    Die seit \acs{PHP} 5 als veraltet erklärte Erweiterung \textit{mysql} wird nicht mehr unterstützt. Dies wird mit Sicherheitsrisiken begründet. So unterstützt
    \textit{mysql} beispielsweise keine \textbf{Prepared Statements}, welche einen wirksamen Schutz gegen \textbf{SQL Injections} bieten. \cite{oracle_mysql_nodate}
    Zudem stehen mit \textit{mysqli} und \textit{PDO} aktuellere Erweiterungen zur Verfügung.
    
    \subsection{ereg}\label{ereg}
    Die Erweiterung \textit{ereg} bietet verschiedene Funktionen für die Nutzung von \textbf{POSIX}-kompatiblen \textbf{Regulären Ausdrücken}. Die Erweiterung wurde
    zugungsten von \textit{\acs{PCRE}} entfernt, da diese unter anderem bessere Unterstützung von Unicode-Zeichen bietet und aktiv weiterentwickelt wird. \cite{popov_php:_2014}

\section{Fazit}
Die Ziele der Weiterentwicklung von \acs{PHP} lassen sich folgendermaßen zusammenfassen:
\begin{itemize}
    \item Erhöhung der Sicherheit
    \item Bessere Verständlichkeit des geschriebenen Quellcodes
\end{itemize}  
Die Erhöhung der Sicherheit wird hauptsächlich durch die Entfernung von Erweiterungen erreicht, die nicht weiterentwickelt werden. Entwickler werden dadurch gezwungen,
diese mit Mitteln zu ersetzen, welche zum Einen auch zukünftig mit Sicherheitsupdates versorgt werden und zum Anderen Features bieten um den Quellcode zusätzlich
zu härten. Auch die Erweiterung von Konzepten der Typsicherheit und die Ersetzung von als Sicherheitsrisiko geltenden Funktionen dient diesem Zweck.\\
Dem Ziel der besseren Verständlichkeit von Quellcode dienen beispielsweise die Vereinheitlichung von Variableninterpretation, die Abschaffung der doppelten Konzepte
der Deklaration von Konstruktoren und die Anpassung der \textit{list()}-Funktion an gewohnte Denkweisen.\\