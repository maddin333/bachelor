%*****************************************
\chapter{Anforderungsanalyse}\label{ch:php7}
%*****************************************
Dieser Abschnitt soll beleuchten, welche Bedingungen PHP in Version 7 gegenüber Version 5 an lauffähige Software stellt. Zudem werden die Änderungen in den Kontext der
zeitlichen Entwicklung gestellt, um Aussagen über Gründe dieser zu treffen.

\section{Abwärtsinkompatible Änderungen}
Änderungen in dieser Kategorie führen in älteren Versionen zu Fehlern oder unerwartetem Verhalten und sind in dieser Umgebung somit nicht lauffähig. Durch diese wird ein 
Wechsel der Umgebung zwingend vorrausgesetzt.

\section{Veraltete Funktionen}
Als veraltet markierte Funktionen sind in der neuen Umgebung zwar noch unterstützt, sollten aber nach Möglichkeit nicht mehr eingesetzt und schnellstmöglich durch geeignete 
Funktionen ersetzt werden, da sie möglicherweise in zukünftigen Versionen entfernt oder verändert werden. Werden diese Funktionen trotzdem eingesetzt, wird eine Warnung 
ausgegeben, die Lauffähigkeit des Programms jedoch nicht beeinflusst.
    \subsection{Implizite Benennung von Konstruktoren}
    Mit der Einführung der objektorientierten Programmierung in PHP 4 wurde festgelegt, dass Funktionen mit dem selben Namen wie die umschließende Klasse implizit als 
    Konstruktor der Klasse erkannt werden. Ein Beispiel zur Implementierung eines Konstruktors nach diesem Prinzip ist in Listing~\ref{lst:php4construct} dargestellt.
    PHP 7 unterstützt diese Notation zwar noch, allerdings wird die, in PHP 5 eingeführte, explizite Benennung mit dem Schlüsselwort \textit{\_\_construct} (siehe 
    Listing~\ref{lst:php5construct}) bevorzugt.
        \begin{lstlisting}[language=php, caption={Beispiel eines impliziten Konstruktors}, label={lst:php4construct}]
            <?php
            class foo {
                function foo($a) {
                    echo("Created instance of class 'foo'");
                }
            }
            ?>
        \end{lstlisting}

        \begin{lstlisting}[language=php, caption={Beispiel eines expliziten Konstruktors}, label={lst:php5construct}]
            <?php
            class foo {
                function __construct($a) {
                    echo("Created instance of class 'foo'");
                }
            }
            ?>
        \end{lstlisting}
    
    \subsection{Statische Aufrufe nicht-statischer Funktionen}

\section{Geänderte Funktionen}
