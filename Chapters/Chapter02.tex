%*****************************************
\chapter{Tätigkeiten}\label{ch:taetigkeiten}
%*****************************************
%\setcounter{figure}{10}
% \begin{flushright}
% \itshape Robert Cialdini, Scott Adams, and Tony Robbins
% \end{flushright}
% \NoCaseChange{Homo Sapiens}
Im Folgenden werden die durchgeführten Tätigkeiten beschrieben, sowie dazugehörige Überlegungen dargelegt.

% Ugly work-around
% Part~\textsc{\ref{pt:showcase}}

% Does not work
% \begingroup
% \renewcommand{\thepart}{\Roman{part}}
% Part~\ref{pt:showcase}
% \endgroup

\section{Migration des Shopsystems zu php7}
Dieses Projekt wurde während meiner Werkstudententätigkeit im Unternehmen begonnen und größtenteils fertiggestellt. 
Da nun der Rollout bevorstand, wurden durch mich hauptsächlich Tests und Bugfixes durchgeführt. 
Diese bezogen sich zum Großteil auf die Hauptfunktionalitäten des Shops und des administrativen Bereichs, sowie auf den Checkout. 
Bugfixes wurden für das Session-Handling geschrieben, das durch die Umstellung des Datenbank-Handlers von \textit{mysql} 
(Diese php-Erweiterung wird mit Version 7 nicht mehr unterstützt) auf \textit{mysqli} noch einige Fehler enthielt. 
Diese fielen bei der Migration nicht direkt auf, da in der lokalen Entwicklungsumgebung nicht auf ein datenbankbasiertes, sondern auf dateibasiertes 
Session-Handling gesetzt wurde. Um den Entwicklungsprozess zu vereinfachen, wurde durch einen externen Entwickler ein Docker-Image bereitgestellt, 
welches durch mich so angepasst wurde, dass ein Debugging mittels der php-Erweiterung \textit{XDebug} auch unter Windows ermöglicht wird. 
Diese Funktionalität wurde im internen Wiki dokumentiert, um anderen Entwicklern die Einrichtung der Entwicklungsumgebung zu erleichtern.
Der Rollout wurde durch einen externen Dienstleister dahingehend vorbereitet, dass Server mit den entsprechenden Softwarepaketen zur Verfügung standen, sodass 
dort nur das Shopsystem aufgespielt werden musste, sowie der Traffic auf diese Server umgeleitet werden musste.
Um möglichst wenige Kunden zu beeinträchtigen und genug Zeit für Tests und einen eventuellen Rollback zu haben, wurde der Rollout früh morgens durchgeführt
und durch meinen Vorgesetzten und mich betreut. Beim anschließenden Testen fielen noch einige fehlerhafte Konfigurationen auf, so zum Beispiel eine 
falsch angegebene Datenbankverbindung. Seitdem diese anfänglichen Probleme behoben sind, läuft das Shopsystem stabil im Produktiveinsatz.


\section{Anpassungen im Versand-Workflow}
Dieses Projekt hat zum Ziel, den Umgang mit verschiedenen Versandarten vom Bestellen bis zum Versenden zu vereinfachen und gleichzeitig flexibler zu gestalten. 
Hierzu wurden neue Versandmodule für das Gambio-System erstellt. Diese bestehen hauptsächlich aus SQL-Anweisungen, die bei der Installation ausgeführt werden 
und Parameter wie zum Beispiel die erlaubten Versandländer in die Datenbank schreiben. Ein Codebeispiel ist in \autoref{lst:shipping} angeführt. 
Zusätzlich wurde ein neues Feld in die administrative Oberfläche der Versandmodule eingeführt, welches es ermöglicht, 
pro Versandzone bzw. Land eine individuelle Benennung der Versandart zu wählen.
\begin{lstlisting} [language=PHP, tabsize=2, caption={Versandmodul Beispiel}, label={lst:shipping}]
function install() {
	xtc_db_query("insert into " . TABLE_CONFIGURATION . " ( configuration_key, configuration_value, configuration_group_id, sort_order, set_function, date_added) VALUES ('MODULE_SHIPPING_STANDARDDE_STATUS', 'True', '6', '0', 'gm_cfg_select_option(array(\'True\', \'False\'), ', now())");
	xtc_db_query("insert into " . TABLE_CONFIGURATION . " ( configuration_key, configuration_value,  configuration_group_id, sort_order, date_added) values ('MODULE_SHIPPING_STANDARDDE_HANDLING', '0', '6', '0', now())");
	for ($i = 1; $i <= $this->num_standardDE; $i++) {
		xtc_db_query("insert into " . TABLE_CONFIGURATION . " (configuration_key, configuration_value, configuration_group_id, sort_order, date_added) values ('MODULE_SHIPPING_STANDARDDE_AFTERBUY_" . $i."', 'Versicherter Standard-Versand', '6', '0', now())");
	}
}

function keys() {
	$keys = array('MODULE_SHIPPING_STANDARDDE_STATUS',
			'MODULE_SHIPPING_STANDARDDE_HANDLING',
			'MODULE_SHIPPING_STANDARDDE_ALLOWED',
			'MODULE_SHIPPING_STANDARDDE_FREEAMOUNT',
			'MODULE_SHIPPING_STANDARDDE_TAX_CLASS',
			'MODULE_SHIPPING_STANDARDDE_ZONE',
			'MODULE_SHIPPING_STANDARDDE_SORT_ORDER');
	
	for ($i = 1; $i <= $this->num_standardDE; $i ++) {
		$keys[count($keys)] = 'MODULE_SHIPPING_STANDARDDE_COUNTRIES_' . $i;
		$keys[count($keys)] = 'MODULE_SHIPPING_STANDARDDE_COST_' . $i;
		$keys[] = 'MODULE_SHIPPING_STANDARDDE_AFTERBUY_' . $i;
	}
	return $keys;
}
\end{lstlisting}

\section{Lokalisation des Shopsystems}
Ziel dieses Projekts ist es, Kunden in Fremdmärkten eine simplere Navigation im Shop zu ermöglichen und die Suchmaschinenrelevanz in diesen Märkten zu erhöhen. 
Bisher sind nur deutsche und englische Sprachversionen des Shops verfügbar, die in einer ersten Umsetzung um zwei weitere Sprachen wichtiger Märkte erweitert werden sollen. 
Hierfür wurde zuerst eine Bestandsaufnahme der bisherigen Übersetzungen vorgenommen, bei der evaluiert wurde, welche Teile des Shops übersetzt werden müssen und welche 
Sprachdateien sich nicht mehr in Benutzung befinden. Von dieser Analyse ausgehend wurden aus diesen Dateien die für die Übersetzung wichtigen Bestandteile 
per \textit{Regular Expression} extrahiert und in das \textit{CSV-Format} überführt, welches von den meisten Übersetzungsbüros gefordert wird. 
Dieses Skript ist in \autoref{lst:languageRegEx} aufgeführt. Eine Herausforderung bestand darin, dass die Sprachdateien in unterschiedlichen Formaten vorlagen. 
\textit{Gambio} arbeitet vorwiegend mit Konstanten, die nach dem Laden der Datei global verfügbar sind (Ein besipielhafter Ausschnitt ist in \autoref{lst:language1} angefügt), 
wohingegen die Templating-Engine Smarty bevorzugt mit, auf das jeweilige Template zugeschnittenen, Arrays arbeitet. 
Für die Übersetzungen wurden Angebote von zwei verschiedenen Übersetzungsbüros angefordert und an jedes ein Auftrag für die Übersetzung in eine Sprache vergeben.
Diese Vorgehensweise sollte uns Erfahrungswerte im Umgang mit den zurückgelieferten Dateien, sowie der Übersetzungsqualität liefern.
Die übersetzten Dateien lagen im CSV-Format vor und mussten wieder möglichst effizient in die ursprünglichen Formate überführt werden.
Hierzu wurde ein Python-Skript geschrieben, welches eine gegebene Ordnerstruktur durchläuft und die Dateien je nach Verzeichnis in die entsprechende Form überträgt.

Eine zusätzliche Herausforderung waren Übersetzungen von Städten und Ländern, die aus Kostengründen nicht an die Übersetzungsbüros gegeben wurden.
Um diese zu übersetzen wurden zwei unterschiedliche Skripte geschrieben. Das Skript für die Übersetzung der Länder \autoref{lst:countriesScript} bedient sich einer freien Datenbank mit Übersetzungen 
aller Länder in unterschiedlichste Sprachen und Dialekte. Um die entsprechenden Übersetzungen zu generieren kann per Kommandozeile die gewünschte Sprache als
Paramter angegeben werden (z.B. "he" für hebräisch). Die Ausgabe erfolgt in einer Datei, die direkt nutzbare SQL-Statements enthält, welche nach der Installation des Sprachpakets 
ausgeführt werden können.

Die Übersetzung der Städte funktioniert nach einem ähnlichen Prinzip, allerdings kommt hier der Open-Source-Webdienst \textit{geonames.org} zum Einsatz, 
dessen API zu einem gegebenen Ort dessen lokale Bezeichnung liefern kann. Ein Beispiel hierfür ist der API-Aufruf in \autoref{lst:geoAPI}, der als
Ergebnis der Abfrage nach der niederländischen Bezeichnung der Stadt Köln die Bezeichnung \textit{Keulen} zurückgibt. Die Ergebnisse müssen allerdings
sorgfältig geprüft werden, da Ortsnamen keine eindeutigen Bezeichner sind.

Nach Tests mit mehreren Sprachen mussten noch einige Bugfixes geschrieben werden. Beispielsweise konnten einige Eingabemasken nur mit aufeinanderfolgenden IDs der
Sprachen umgehen, an anderen Stellen wurden Masken schon mit drei Sprachen zu unübersichtlich und mussten überarbeitet werden.

Das Projekt ist noch nicht abgeschlossen und wird nach dem Praktikum fertiggestellt.

\lstset{literate=%
	{Ö}{{\"O}}1
	{Ä}{{\"A}}1
	{Ü}{{\"U}}1
	{ß}{{\ss}}1
	{ü}{{\"u}}1
	{ä}{{\"a}}1
	{ö}{{\"o}}1
	{~}{{\textasciitilde}}1
}
\begin{lstlisting} [language=PHP, caption={Sprachdefinition mittels Konstanten}, label={lst:language1}]
define('GM_WISHLIST_NOTHING_CHECKED', 'Sie haben keine Artikel ausgewählt, die in den Warenkorb gelegt werden sollen!');
define('JS_ERROR_CONDITIONS_NOT_ACCEPTED_AGB', 'Sofern Sie unsere Allgemeinen Geschäftsbedingungen nicht akzeptieren,\n können wir Ihre Bestellung leider nicht entgegennehmen! \n\n');
define('JS_ERROR_CONDITIONS_NOT_ACCEPTED_WITHDRAWAL', 'Sofern Sie unser Widerrufsrecht nicht akzeptieren,\n können wir Ihre Bestellung leider nicht entgegennehmen! \n\n');
\end{lstlisting}
\begin{lstlisting} [language=PHP, caption={Sprachdefinition mittels Arrays}, label={lst:language2}]
$t_language_text_section_content_array = array(
'page_not_found' => 'Leider konnten wir die von Ihnen gewünschte Veranstaltung nicht finden.',
'page_not_found_introtext' => 'Die von Ihnen gewünschte Seite konnte nicht gefunden werden. Entweder ist die Veranstaltung nicht mehr verfügbar oder wurde auf eine andere Seite verschoben.',
'try_following' => 'Dies könnte Ihnen helfen Ihre Wunschveranstaltung zu finden:',
'navigate_to_frontpage' => 'Besuchen Sie unsere <a href="/">Startseite</a>',
'browse_events' => 'Stöbern Sie in unseren <a href="upcomming_products.php">Veranstaltungen von A bis Z</a>',
'use_search_functionality' => 'Nutzen Sie die Suchfunktion:',
'navigate_to_categories' => 'Navigieren Sie zu einer der Kategorien',
'navigate_to' => 'Navigieren Sie zu'
);
\end{lstlisting}

\begin{lstlisting} [language=PHP, caption={Extrahieren der Übersetzungen}, label={lst:languageRegEx}]
foreach(glob("*.php") as $filename){
  $match = [];
  $langString = "";
  if (is_file($filename) && $filename != 'langExtract.php'){
    //Alle Definitionen von Konstaten
    if (preg_match_all("/define\W*\(\W*'([^']*)\W*'([^']*)/", file_get_contents($filename), $match)){
      break;
	//Alle Definitionen innerhalb des Arrays
	} else if (preg_match_all("/\W*'([^']*)\W*'([^']*)/", file_get_contents($filename), $match)) {
	  break;
	} else {
	  print('Fehler in Datei ' . $filename);
	}

	for($i = 0; $i < sizeof($match[1]); $i++){
	  $langString .= "\"" . $match[1][$i] . "\";" . "\"" . html_entity_decode($match[2][$i], ENT_HTML401 ,'cp1252') . "\";\n";
	}

	$fp = fopen( "csv/" . $filename . ".csv", "w");
	fwrite($fp, $langString);
  }
}
\end{lstlisting}

\begin{lstlisting} [language=PHP, caption={Länderübersetzungen}, label={lst:countriesScript}]
$countries = $argv;
unset($countries['0']);
$csv = array_map('str_getcsv', file('IP2LOCATION-COUNTRY-MULTILINGUAL.CSV'));
$fp = fopen('countries-lang-' . implode(", ", $countries) . '.sql', 'a');
foreach($countries as $lang_to_insert) {
	fwrite($fp, "SET @var = (SELECT languages_id FROM languages WHERE `code` = '" . $lang_to_insert . "');");
	foreach ($csv as $key => $value){
		if ($lang_to_insert == strtolower($value[0])) {
			fwrite($fp, "UPDATE countries_translations 
				SET countries_name = '" . str_replace("'", "''", $value[5]) ."'
				WHERE language_id = @var AND countries_iso_code_2 = '". $value[2] ."';\n");
		}
	}
}
\end{lstlisting}

\begin{lstlisting} [language=PHP, caption={API-Aufruf geonames.org}, label={lst:geoAPI}]
$url = 	'http://api.geonames.org/search?q=K%C3%B6ln&country=DE&lang=nl&username=demo';
$response = file_get_contents($url);
\end{lstlisting}

\section{Verschiedene Tätigkeiten} % \ensuremath{\NoCaseChange{\mathbb{ZNR}}}
\subsection{Austausch des API-Keys im Zahlungsmodul}
Der Zahlungsdienstleister \textit{Klarna} empfiehlt, den zur Validierung eingesetzten API-Key aus Sicherheitsgründen mindestens ein mal jährlich auszutauschen. Um diesen Anforderungen gerecht zu werden, wurde über die Administrationsoberfläche ein neuer API-Key erstellt und in der lokalen Umgebung erfolgreich getestet. Anschließend wurde dieser in das Shop-System eingepflegt und im internen Wiki dokumentiert.
\subsection{Bugfixes in der Produktpflegetabelle}
In der bestehenden Produktpflegetabelle können aktive Produkte bearbeitet werden. Dies umfasst die Zuordnung von Saalplänen, die Bearbeitung von Übersetzungen, SEO-Keywords und Eigenschaften der Produkte, sowie die Erstellung von Filterkategorien, durch welche Produktkategorien sortiert werden können. Durch den großen Funktionsumfang kommt es immer wieder zu kleineren Fehlern die behoben werden müssen.
\subsubsection{Zuordnung von Eigenschaften zu Produkten}
Durch diesen Bug war es nicht mehr möglich, einem Produkt Eigenschaften zuzuordnen. Diese sind Voraussetzung um Filter erstellen zu können.
\subsubsection{Abspeichern von Hinweisen nicht mehr möglich}
Dieser Bug bewirkte, dass der Speichern-Button für Hinweise bezüglich der Platzwahl nicht mehr angezeigt wurde. Ursache für diesen Bug war, dass das PHP-Skript durch eine Exception abgebrochen wurde, bevor der Button geladen wurde und wurde durch einen früheren Fix ausgelöst, wodurch die betreffende Funktion mit einem leeren Array aufgerufen wurde. Die alte Funktionalität konnte durch kurze Recherche in der Versionsgeschichte der betreffenden Datei wiederhergestellt werden.
\subsection{Projektmanagement Produktpflegetabelle}
Für die Entwicklung einer neuen Produktpflegetabelle übernahm ich die Konzeption, sowie das Projektmanagement. Hierbei spielte vor allem die Benutzerfreundlichkeit 
eine große Rolle, da die aktuell eingesetzte Version eine initiale Ladezeit von über einer Minute auweist. Um dem entgegenzuwirken und auch zukünftig auszuschließen,
wurde die Tabellenstruktur angepasst, wodurch ein \textit{Lazyloading} (Nachladen von Inhalten erst auf Anfrage) möglich wurde. Ebenso wurde eine Pagination 
eingeführt, um auch bei einer deutlichen Steigerung des Angebots nachteilige Auswirkungen auf die Geschwindigkeit auszuschließen. Eine spannende und besondere 
Herausforderung bestand darin, dass sich das unterliegende System (eine modulare Umgebung, basierend auf \textit{Kotlin} und \textit{VueJS}) noch in der Entwicklung
befindet und ich bis zum Projektstart keine Erfahrungswerte damit sammeln konnte.
\subsection{Systemadministration}
Als Tätigkeiten in der Systemadministration sind vor allem die Konfiguration der Cloud-basierten Telefonanlage, vor allem die Einrichtung neuer Gegenstellen
sowie die Behandlung eines Problems nach einem fehlerhaften Update, das durch ein Zurücksetzen der Geräte gelöst werden konnte, sowie die Konfiguration
der Firewall. Diese ist sehr restriktiv eingestellt, so dass einige Webseiten und Services nicht auf Anhieb funktionieren und zuerst eine Ausnahme 
in der Weboberfläche hinzugefügt werden muss.




%*****************************************
%*****************************************
%*****************************************
%*****************************************
%*****************************************
